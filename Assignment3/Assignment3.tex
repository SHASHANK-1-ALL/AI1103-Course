\documentclass[journal,12pt,twocolumn]{IEEEtran}

\usepackage{setspace}
\usepackage{gensymb}
\singlespacing
\usepackage[cmex10]{amsmath}

\usepackage{amsthm}

\usepackage{mathrsfs}
\usepackage{txfonts}
\usepackage{stfloats}
\usepackage{bm}
\usepackage{cite}
\usepackage{cases}
\usepackage{subfig}

\usepackage{longtable}
\usepackage{multirow}

\usepackage{enumitem}
\usepackage{mathtools}
\usepackage{steinmetz}
\usepackage{tikz}
\usepackage{circuitikz}
\usepackage{verbatim}
\usepackage{tfrupee}
\usepackage[breaklinks=true]{hyperref}
\usepackage{graphicx}
\usepackage{tkz-euclide}

\usetikzlibrary{calc,math}
\usepackage{listings}
    \usepackage{color}                                            %%
    \usepackage{array}                                            %%
    \usepackage{longtable}                                        %%
    \usepackage{calc}                                             %%
    \usepackage{multirow}                                         %%
    \usepackage{hhline}                                           %%
    \usepackage{ifthen}                                           %%
    \usepackage{lscape}     
\usepackage{multicol}
\usepackage{chngcntr}

\DeclareMathOperator*{\Res}{Res}

\renewcommand\thesection{\arabic{section}}
\renewcommand\thesubsection{\thesection.\arabic{subsection}}
\renewcommand\thesubsubsection{\thesubsection.\arabic{subsubsection}}

\renewcommand\thesectiondis{\arabic{section}}
\renewcommand\thesubsectiondis{\thesectiondis.\arabic{subsection}}
\renewcommand\thesubsubsectiondis{\thesubsectiondis.\arabic{subsubsection}}


\hyphenation{op-tical net-works semi-conduc-tor}
\def\inputGnumericTable{}                                 %%

\lstset{
%language=C,
frame=single, 
breaklines=true,
columns=fullflexible
}
\begin{document}

\newcommand{\BEQA}{\begin{eqnarray}}
\newcommand{\EEQA}{\end{eqnarray}}
\newcommand{\define}{\stackrel{\triangle}{=}}
\bibliographystyle{IEEEtran}
\raggedbottom
\setlength{\parindent}{0pt}
\providecommand{\mbf}{\mathbf}
\providecommand{\pr}[1]{\ensuremath{\Pr\left(#1\right)}}
\providecommand{\qfunc}[1]{\ensuremath{Q\left(#1\right)}}
\providecommand{\sbrak}[1]{\ensuremath{{}\left[#1\right]}}
\providecommand{\lsbrak}[1]{\ensuremath{{}\left[#1\right.}}
\providecommand{\rsbrak}[1]{\ensuremath{{}\left.#1\right]}}
\providecommand{\brak}[1]{\ensuremath{\left(#1\right)}}
\providecommand{\lbrak}[1]{\ensuremath{\left(#1\right.}}
\providecommand{\rbrak}[1]{\ensuremath{\left.#1\right)}}
\providecommand{\cbrak}[1]{\ensuremath{\left\{#1\right\}}}
\providecommand{\lcbrak}[1]{\ensuremath{\left\{#1\right.}}
\providecommand{\rcbrak}[1]{\ensuremath{\left.#1\right\}}}
\theoremstyle{remark}
\newtheorem{rem}{Remark}
\newcommand{\sgn}{\mathop{\mathrm{sgn}}}
\providecommand{\abs}[1]{\vert#1\vert}
\providecommand{\res}[1]{\Res\displaylimits_{#1}} 
\providecommand{\norm}[1]{\lVert#1\rVert}
%\providecommand{\norm}[1]{\lVert#1\rVert}
\providecommand{\mtx}[1]{\mathbf{#1}}
\providecommand{\mean}[1]{E[ #1 ]}
\providecommand{\fourier}{\overset{\mathcal{F}}{ \rightleftharpoons}}
%\providecommand{\hilbert}{\overset{\mathcal{H}}{ \rightleftharpoons}}
\providecommand{\system}{\overset{\mathcal{H}}{ \longleftrightarrow}}
	%\newcommand{\solution}[2]{\textbf{Solution:}{#1}}
\newcommand{\solution}{\noindent \textbf{Solution: }}
\newcommand{\cosec}{\,\text{cosec}\,}
\providecommand{\dec}[2]{\ensuremath{\overset{#1}{\underset{#2}{\gtrless}}}}
\newcommand{\myvec}[1]{\ensuremath{\begin{pmatrix}#1\end{pmatrix}}}
\newcommand{\mydet}[1]{\ensuremath{\begin{vmatrix}#1\end{vmatrix}}}
\numberwithin{equation}{subsection}
\makeatletter
\@addtoreset{figure}{problem}
\makeatother
\let\StandardTheFigure\thefigure
\let\vec\mathbf
\renewcommand{\thefigure}{\theproblem}
\def\putbox#1#2#3{\makebox[0in][l]{\makebox[#1][l]{}\raisebox{\baselineskip}[0in][0in]{\raisebox{#2}[0in][0in]{#3}}}}
     \def\rightbox#1{\makebox[0in][r]{#1}}
     \def\centbox#1{\makebox[0in]{#1}}
     \def\topbox#1{\raisebox{-\baselineskip}[0in][0in]{#1}}
     \def\midbox#1{\raisebox{-0.5\baselineskip}[0in][0in]{#1}}
\vspace{3cm}
\title{\textbf{AI1103-Assignment 3}}
\author{Name: Shashank Shanbhag, Roll Number: CS20BTECH11061 }
\maketitle
\newpage
\bigskip
\renewcommand{\thefigure}{\theenumi}
\renewcommand{\thetable}{\theenumi}

Download all python codes from \begin{lstlisting}
https://github.com/SHASHANK-1-ALL/AI1103-Assignment-3/blob/main/Assignment3.py
\end{lstlisting}
%
and latex-tikz codes from 
%
\begin{lstlisting}
https://github.com/SHASHANK-1-ALL/AI1103-Assignment-3/blob/main/Assignment3.tex
\end{lstlisting}
\section*{\textbf{Question}}
The lifetime of two brands of bulbs X and Y are exponentially distributed with the mean life of 100 hours. Bulb X is switched on 15 hours after bulb Y has been switched on. The probability that bulb X fails before bulb Y is 

\begin{enumerate}[label=(\Alph*)]
    \item $\frac{15}{100}$ \\
    \item $\frac{1}{2}$ \\
    \item $\frac{85}{100}$ \\
    \item 0 
  \end{enumerate}
  
\section*{\textbf{Solution}}
Let X and Y be exponential random variables which represent the lifetime of bulbs X and Y respectively, both with mean = 100.\\

Using memorylessness property for exponential distribution, which states that :\\
\emph{An exponentially distributed random variable T obeys the relation}
\begin{align}
  \pr{T>t+s |T>s} = \pr{T>t}
\end{align} 
\emph{where $s,t\geq 0$}\\

Proof : Using Complementary cumulative distributive function, we get
\begin{align}
   \pr{T>t+s |T>s} =& \frac{\pr{T>t+s , T>s}}{\pr{T>s}} \\
   =& \frac{\pr{T>t+s}}{\pr{T>s}} \\
   =& \frac{e^{-\lambda(t+s)}}{e^{-\lambda s}}\\
   =&  e^{-\lambda t}\\
   =& \pr{T>t}
\end{align}

Probability that bulb X fails before bulb Y given that bulb Y was functioning when bulb X was switched on 
\begin{align}
  \pr{Y>X+15 | Y \geq 15}
  =& \pr{Y>X}
\end{align}

For both X and Y,
\begin{equation}
\lambda= \frac{1}{100} = 0.01
\end{equation}
\\
Probability distribution function of exponential random variables is given by : 

For x,y $\geq$ 0
\begin{align}
    \texorpdfstring{f\textsubscript{X}}{f X}(x) = \lambda e^{-\lambda x}\\
    \texorpdfstring{f\textsubscript{Y}}{f Y}(y) = \lambda e^{-\lambda y}
\end{align}

Cumulative distribution function of exponential random variables is given by : 

For x $\geq$ 0
\begin{align}
    \texorpdfstring{F\textsubscript{X}}{F X}(x) = 1- e^{-\lambda x}\\
    \texorpdfstring{F\textsubscript{Y}}{F Y}(x) = 1- e^{-\lambda x}
\end{align}

\begin{align}
    \pr{Y>X} =& \int_{-\infty}^{\infty}  \texorpdfstring{F\textsubscript{Y}}{F Y}(x) \texorpdfstring{f\textsubscript{X}}{f X}(x)  dx\\
    =& \int_{0}^{\infty} ( 1- e^{-\lambda x}) \lambda e^{-\lambda x}\\
    =&  \lambda \brak{\frac{1}{2\lambda} e^{-2\lambda x} - \frac{1}{\lambda} e^{-\lambda x}} \Bigg|_0^\infty\\
     =& \brak{\frac{1}{2} e^{-2\lambda x} - e^{-\lambda x}} \Bigg|_0^\infty\\
    =& \brak{\frac{1}{2} e^{-0.02x} - e^{-0.01x}} \Bigg|_0^\infty\\
    =& \frac{1}{2} = 0.5
\end{align}

 $\therefore$The answer is option (b) \large $\frac{1}{2}$.

\end{document}
