\documentclass[journal,12pt,twocolumn]{IEEEtran}

\usepackage{setspace}
\usepackage{gensymb}
\singlespacing
\usepackage[cmex10]{amsmath}

\usepackage{amsthm}

\usepackage{mathrsfs}
\usepackage{txfonts}
\usepackage{stfloats}
\usepackage{bm}
\usepackage{cite}
\usepackage{cases}
\usepackage{subfig}

\usepackage{longtable}
\usepackage{multirow}

\usepackage{enumitem}
\usepackage{mathtools}
\usepackage{steinmetz}
\usepackage{tikz}
\usepackage{circuitikz}
\usepackage{verbatim}
\usepackage{tfrupee}
\usepackage[breaklinks=true]{hyperref}
\usepackage{graphicx}
\usepackage{tkz-euclide}
\newcommand*{\Comb}[2]{{}^{#1}C_{#2}}%

\usetikzlibrary{calc,math}
\usepackage{listings}
    \usepackage{color}                                            %%
    \usepackage{array}                                            %%
    \usepackage{longtable}                                        %%
    \usepackage{calc}                                             %%
    \usepackage{multirow}                                         %%
    \usepackage{hhline}                                           %%
    \usepackage{ifthen}                                           %%
    \usepackage{lscape}     
\usepackage{multicol}
\usepackage{chngcntr}

\DeclareMathOperator*{\Res}{Res}

\renewcommand\thesection{\arabic{section}}
\renewcommand\thesubsection{\thesection.\arabic{subsection}}
\renewcommand\thesubsubsection{\thesubsection.\arabic{subsubsection}}

\renewcommand\thesectiondis{\arabic{section}}
\renewcommand\thesubsectiondis{\thesectiondis.\arabic{subsection}}
\renewcommand\thesubsubsectiondis{\thesubsectiondis.\arabic{subsubsection}}


\hyphenation{op-tical net-works semi-conduc-tor}
\def\inputGnumericTable{}                                 %%

\lstset{
%language=C,
frame=single, 
breaklines=true,
columns=fullflexible
}
\begin{document}

\newcommand{\BEQA}{\begin{eqnarray}}
\newcommand{\EEQA}{\end{eqnarray}}
\newcommand{\define}{\stackrel{\triangle}{=}}
\bibliographystyle{IEEEtran}
\raggedbottom
\setlength{\parindent}{0pt}
\providecommand{\mbf}{\mathbf}
\providecommand{\pr}[1]{\ensuremath{\Pr\left(#1\right)}}
\providecommand{\qfunc}[1]{\ensuremath{Q\left(#1\right)}}
\providecommand{\sbrak}[1]{\ensuremath{{}\left[#1\right]}}
\providecommand{\lsbrak}[1]{\ensuremath{{}\left[#1\right.}}
\providecommand{\rsbrak}[1]{\ensuremath{{}\left.#1\right]}}
\providecommand{\brak}[1]{\ensuremath{\left(#1\right)}}
\providecommand{\lbrak}[1]{\ensuremath{\left(#1\right.}}
\providecommand{\rbrak}[1]{\ensuremath{\left.#1\right)}}
\providecommand{\cbrak}[1]{\ensuremath{\left\{#1\right\}}}
\providecommand{\lcbrak}[1]{\ensuremath{\left\{#1\right.}}
\providecommand{\rcbrak}[1]{\ensuremath{\left.#1\right\}}}
\theoremstyle{remark}
\newtheorem{rem}{Remark}
\newcommand{\sgn}{\mathop{\mathrm{sgn}}}
\providecommand{\abs}[1]{\vert#1\vert}
\providecommand{\res}[1]{\Res\displaylimits_{#1}} 
\providecommand{\norm}[1]{\lVert#1\rVert}
%\providecommand{\norm}[1]{\lVert#1\rVert}
\providecommand{\mtx}[1]{\mathbf{#1}}
\providecommand{\mean}[1]{E[ #1 ]}
\providecommand{\fourier}{\overset{\mathcal{F}}{ \rightleftharpoons}}
%\providecommand{\hilbert}{\overset{\mathcal{H}}{ \rightleftharpoons}}
\providecommand{\system}{\overset{\mathcal{H}}{ \longleftrightarrow}}
	%\newcommand{\solution}[2]{\textbf{Solution:}{#1}}
\newcommand{\solution}{\noindent \textbf{Solution: }}
\newcommand{\cosec}{\,\text{cosec}\,}
\providecommand{\dec}[2]{\ensuremath{\overset{#1}{\underset{#2}{\gtrless}}}}
\newcommand{\myvec}[1]{\ensuremath{\begin{pmatrix}#1\end{pmatrix}}}
\newcommand{\mydet}[1]{\ensuremath{\begin{vmatrix}#1\end{vmatrix}}}
\numberwithin{equation}{subsection}
\makeatletter
\@addtoreset{figure}{problem}
\makeatother
\let\StandardTheFigure\thefigure
\let\vec\mathbf
\renewcommand{\thefigure}{\theproblem}
\def\putbox#1#2#3{\makebox[0in][l]{\makebox[#1][l]{}\raisebox{\baselineskip}[0in][0in]{\raisebox{#2}[0in][0in]{#3}}}}
     \def\rightbox#1{\makebox[0in][r]{#1}}
     \def\centbox#1{\makebox[0in]{#1}}
     \def\topbox#1{\raisebox{-\baselineskip}[0in][0in]{#1}}
     \def\midbox#1{\raisebox{-0.5\baselineskip}[0in][0in]{#1}}
\vspace{3cm}
\title{\textbf{AI1103-Assignment 5}}
\author{Name: Shashank Shanbhag, Roll Number: CS20BTECH11061 }
\maketitle
\newpage
\bigskip
\renewcommand{\thefigure}{\theenumi}
\renewcommand{\thetable}{\theenumi}

Download all python codes from
\begin{lstlisting}
https://github.com/SHASHANK-1-ALL/AI1103-Assignment-5/blob/main/Assignment5.py
\end{lstlisting}
%
and latex-tikz codes from 
%
\begin{lstlisting}
https://github.com/SHASHANK-1-ALL/AI1103-Assignment-5/blob/main/Assignment5.tex
\end{lstlisting}
\section*{\textbf{Question}}
A box contains 2 washers, 3 nuts and 4 bolts. Items are drawn from the box at random one at a time without replacement. The probability of drawing 2 washers first followed by 3 nuts and subsequently 4 bolts is

\begin{multicols}{4}
    \begin{enumerate}[label=(\Alph*)]
        \item \large$\frac{2}{315}$
        \item \large$\frac{1}{630}$
        \item \large$\frac{1}{1260}$
        \item \large$\frac{1}{2520}$
    \end{enumerate}
\end{multicols}

\section*{\textbf{solution}}
Let $X\in\{0,1,2\}$ be the random variable such that X=0 represents that we draw 2 washers, X=1 represents that we draw 3 nuts and X=2 represents that we draw 4 bolts, continuously without replacement. \\

Total number of objects :
\begin{equation}
    N = 2+ 3+ 4 = 9
\end{equation}

Probability of occurrence of X=0 :
\begin{align}
    \pr{X=0} =& \dfrac{\Comb{2}{2}}{\Comb{9}{2}} \\
     =& \dfrac{1}{36}    
\end{align}

Total number of objects after occurrence of X=0 :
\begin{equation}
    N = 3+ 4 = 7
\end{equation}

Probability of occurrence of X=1 given that X=0 has already occurred :
\begin{align}
    \pr{X=1|X=0} =& \dfrac{\Comb{3}{3}}{\Comb{7}{3}}\\
   =& \dfrac{1}{35} 
\end{align}

Total number of objects after occurrence of X=0 and X=1 :
\begin{equation}
    N = 4 
\end{equation}

Probability of occurrence of X=2 given that X=0 and X=1 has already occurred :
\begin{align}
    \pr{X=2|(X=0,X=1)} =&\dfrac{\Comb{4}{4}}{\Comb{4}{4}} \\
     =& 1
\end{align}

Using Multiplication law of probability,\\ Required probability is given by :
\begin{multline}
    \pr{X=0 , X=1 , X=2} \\
      = \pr{X=0}\times \pr{X=1|X=0}\\\times \pr{X=2|(X=0,X=1)}
    \end{multline}
    
\begin{align}
\implies\pr{X=0 , X=1 , X=2} =& \frac{1}{36}\times\frac{1}{35}\times1 \\
=& \frac{1}{1260}
\end{align}

$\therefore$ The correct option is (C) \large $\frac{1}{1260}$.

\end{document}
