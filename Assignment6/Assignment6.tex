\documentclass[journal,12pt,twocolumn]{IEEEtran}

\usepackage{setspace}
\usepackage{gensymb}
\singlespacing
\usepackage[cmex10]{amsmath}

\usepackage{amsthm}

\usepackage{mathrsfs}
\usepackage{txfonts}
\usepackage{stfloats}
\usepackage{bm}
\usepackage{cite}
\usepackage{cases}
\usepackage{subfig}
\usepackage{lipsum} 

\usepackage{longtable}
\usepackage{multirow}

\usepackage{enumitem}
\usepackage{mathtools}
\usepackage{steinmetz}
\usepackage{tikz}
\usepackage{circuitikz}
\usepackage{verbatim}
\usepackage{tfrupee}
\usepackage[breaklinks=true]{hyperref}
\usepackage{graphicx}
\usepackage{tkz-euclide}
\newcommand*{\Comb}[2]{{}^{#1}C_{#2}}%

\newcommand{\tikzAngleOfLine}{\tikz@AngleOfLine}
  \def\tikz@AngleOfLine(#1)(#2)#3{%
  \pgfmathanglebetweenpoints{%
    \pgfpointanchor{#1}{center}}{%
    \pgfpointanchor{#2}{center}}
  \pgfmathsetmacro{#3}{\pgfmathresult}%
  }

\usetikzlibrary{calc,math}
\usepackage{listings}
    \usepackage{color}                                            %%
    \usepackage{array}                                            %%
    \usepackage{longtable}                                        %%
    \usepackage{calc}                                             %%
    \usepackage{multirow}                                         %%
    \usepackage{hhline}                                           %%
    \usepackage{ifthen}                                           %%
    \usepackage{lscape}     
\usepackage{multicol}
\usepackage{chngcntr}

\DeclareMathOperator*{\Res}{Res}

\renewcommand\thesection{\arabic{section}}
\renewcommand\thesubsection{\thesection.\arabic{subsection}}
\renewcommand\thesubsubsection{\thesubsection.\arabic{subsubsection}}

\renewcommand\thesectiondis{\arabic{section}}
\renewcommand\thesubsectiondis{\thesectiondis.\arabic{subsection}}
\renewcommand\thesubsubsectiondis{\thesubsectiondis.\arabic{subsubsection}}
\newcommand{\Int}{\int\limits}
\usepackage{tikz}

\hyphenation{op-tical net-works semi-conduc-tor}
\def\inputGnumericTable{}                                 %%

\lstset{
%language=C,
frame=single, 
breaklines=true,
columns=fullflexible
}
\begin{document}

\newcommand{\BEQA}{\begin{eqnarray}}
\newcommand{\EEQA}{\end{eqnarray}}
\newcommand{\define}{\stackrel{\triangle}{=}}
\bibliographystyle{IEEEtran}
\raggedbottom
\setlength{\parindent}{0pt}
\providecommand{\mbf}{\mathbf}
\providecommand{\pr}[1]{\ensuremath{\Pr\left(#1\right)}}
\providecommand{\qfunc}[1]{\ensuremath{Q\left(#1\right)}}
\providecommand{\sbrak}[1]{\ensuremath{{}\left[#1\right]}}
\providecommand{\lsbrak}[1]{\ensuremath{{}\left[#1\right.}}
\providecommand{\rsbrak}[1]{\ensuremath{{}\left.#1\right]}}
\providecommand{\brak}[1]{\ensuremath{\left(#1\right)}}
\providecommand{\lbrak}[1]{\ensuremath{\left(#1\right.}}
\providecommand{\rbrak}[1]{\ensuremath{\left.#1\right)}}
\providecommand{\cbrak}[1]{\ensuremath{\left\{#1\right\}}}
\providecommand{\lcbrak}[1]{\ensuremath{\left\{#1\right.}}
\providecommand{\rcbrak}[1]{\ensuremath{\left.#1\right\}}}
\theoremstyle{remark}
\newtheorem{rem}{Remark}
\newcommand{\sgn}{\mathop{\mathrm{sgn}}}
\providecommand{\abs}[1]{\vert#1\vert}
\providecommand{\res}[1]{\Res\displaylimits_{#1}} 
\providecommand{\norm}[1]{\lVert#1\rVert}
%\providecommand{\norm}[1]{\lVert#1\rVert}
\providecommand{\mtx}[1]{\mathbf{#1}}
\providecommand{\mean}[1]{E[ #1 ]}
\providecommand{\fourier}{\overset{\mathcal{F}}{ \rightleftharpoons}}
%\providecommand{\hilbert}{\overset{\mathcal{H}}{ \rightleftharpoons}}
\providecommand{\system}{\overset{\mathcal{H}}{ \longleftrightarrow}}
	%\newcommand{\solution}[2]{\textbf{Solution:}{#1}}
\newcommand{\solution}{\noindent \textbf{Solution: }}
\newcommand{\cosec}{\,\text{cosec}\,}
\providecommand{\dec}[2]{\ensuremath{\overset{#1}{\underset{#2}{\gtrless}}}}
\newcommand{\myvec}[1]{\ensuremath{\begin{pmatrix}#1\end{pmatrix}}}
\newcommand{\mydet}[1]{\ensuremath{\begin{vmatrix}#1\end{vmatrix}}}
\numberwithin{equation}{subsection}
\makeatletter
\@addtoreset{figure}{problem}
\makeatother
\let\StandardTheFigure\thefigure
\let\vec\mathbf
\renewcommand{\thefigure}{\theproblem}
\def\putbox#1#2#3{\makebox[0in][l]{\makebox[#1][l]{}\raisebox{\baselineskip}[0in][0in]{\raisebox{#2}[0in][0in]{#3}}}}
     \def\rightbox#1{\makebox[0in][r]{#1}}
     \def\centbox#1{\makebox[0in]{#1}}
     \def\topbox#1{\raisebox{-\baselineskip}[0in][0in]{#1}}
     \def\midbox#1{\raisebox{-0.5\baselineskip}[0in][0in]{#1}}
\vspace{3cm}
\title{\textbf{AI1103-Assignment 6}}
\author{Name: Shashank Shanbhag, Roll Number: CS20BTECH11061 }
\maketitle
\newpage
\bigskip
\renewcommand{\thefigure}{\theenumi}
\renewcommand{\thetable}{\theenumi}

    
Download all python codes from
\begin{lstlisting}
https://github.com/SHASHANK-1-ALL/AI1103-ASSIGNMENT-6/blob/main/Assignment6.py
\end{lstlisting}
%
and latex-tikz codes from 
%
\begin{lstlisting}
https://github.com/SHASHANK-1-ALL/AI1103-ASSIGNMENT-6/blob/main/Assignment6.tex
\end{lstlisting}
\section*{\textbf{Question}}
A point is chosen at random from a circular disc shown below. What is the probability that the point lies in the sector OAB?\\

\begin{tikzpicture}[
    colorstyle/.style={
       circle, draw=black,fill=black,
       thick, inner sep=0pt, minimum size=2 mm,
       outer sep=0pt
        },
    scale=2]
\draw (0,0) circle (2cm);
\node at (0,0) [colorstyle,label=below:O]{};
\node at (1,1.732) [colorstyle,label=above:A]{};
\node at (1.732,1) [colorstyle,label=above right:B]{};
\draw (0,0) -- (1,1.732);
\draw (0,0) -- (1.732,1);
\coordinate (O) at (0,0);
\coordinate (A) at (1,1.732);
\coordinate (B) at (1.732,1);
\tikzAngleOfLine(O)(B){\AngleStart}
    \tikzAngleOfLine(O)(A){\AngleEnd}
    \draw[red,<->] (O)+(\AngleStart:1cm) arc (\AngleStart:\AngleEnd:1 cm);
    \node[circle,fill=green] at ($(O)+({(\AngleStart+\AngleEnd)/2}:1.5 cm)$) {x};
\end{tikzpicture}\\

( where $\angle$AOB = x radians )

\begin{multicols}{2}
    \begin{enumerate}
        \item $\frac{2x}{\pi}$
        \item $\frac{x}{\pi}$
        \item $\frac{x}{2\pi}$
        \item $\frac{x}{4\pi}$
    \end{enumerate}
\end{multicols}

\section*{\textbf{solution}}
Let $X \in \{0,1\}$ be a random variable such that X=0 means we choose a point lying in sector OAB and X=1 means that we choose a point lying outside sector OAB and inside the circle.\\

Area of a sector subtending an angle $\theta$ at the centre of circle with radius a is given by :
\begin{equation}
    A = \frac{1}{2}a^2\theta
\end{equation}
where $\theta$ is in radians.\\

Let the radius of circle shown in figure be r. It is given that  sector  OAB subtends an angle of x radians at the centre of the circle.\\

Probability that the chosen point lies in sector OAB is:
\begin{align}
    \pr{X=0} =& \frac{\text{Area of sector OAB}}{\text{Area of circle}}\\
       =& \frac{\frac{1}{2} r^2 x}{\pi r^2}\\
       =& \frac{x}{2\pi}
\end{align}

$\therefore$The correct answer is \textbf{option (3)} $\frac{x}{2\pi}$.

\section*{\textbf{alternate solution}}
The joint pdf is given by:
\begin{equation}
 \texorpdfstring{f\textsubscript{r$\theta$}}{f r $\theta$}(r,\theta)= \begin{cases}
                        \dfrac{r}{\pi R^2}  & \text{if 0 $<$ r $<$ R , 0 $<$ $\theta$ $<$ 2$\pi$ }\\
                        0  & \text{otherwise}
                        \end{cases}
\end{equation}

Let A $\equiv$ (R,  $\theta _2$) and B $\equiv$ (R,  $\theta _1$). \\
Hence,
\begin{equation}
(\theta _2 - \theta _1)= x    
\end{equation}

We want $\theta$ $\in$ ($\theta _1$ , $\theta _2$) and r $\in$ (0,R) for point to lie in the sector.
Let the point to be chosen be (r, $\theta$).\\

So, Required probability is:
\begin{align}
 \nonumber  \pr{\theta_1<\theta<\theta_2 , 0<r<R}\\
    =& \Int_{\theta_1}^{\theta_2} \Int_{0}^{R} \dfrac{r}{\pi R^2} \,dr\,d\theta \displaybreak \\
    =& \Int_{\theta_1}^{\theta_2} \dfrac{1}{\pi R^2} \dfrac{r^2}{2} \Bigg|_0^R \\
    =& \Int_{\theta_1}^{\theta_2} \dfrac{R^2}{2\pi R^2} \,d\theta   \\
    =& \Int_{\theta_1}^{\theta_2} \dfrac{1}{2\pi} \,d\theta\\
    =& \dfrac{\theta}{2\pi} \Bigg|_{\theta_1}^{\theta_2} \\
    =& \dfrac{\theta_2 - \theta_1}{2\pi} \\
    =& \dfrac{x}{2\pi}
\end{align}
    
$\therefore$The correct answer is \textbf{option (3)} \Large $\frac{x}{2\pi}$.

\end{document}