\documentclass[journal,12pt,twocolumn]{IEEEtran}

\usepackage{setspace}
\usepackage{gensymb}
\singlespacing
\usepackage[cmex10]{amsmath}

\usepackage{amsthm}

\usepackage{mathrsfs}
\usepackage{txfonts}
\usepackage{stfloats}
\usepackage{bm}
\usepackage{cite}
\usepackage{cases}
\usepackage{subfig}
\usepackage{lipsum} 

\usepackage{longtable}
\usepackage{multirow}

\usepackage{enumitem}
\usepackage{mathtools}
\usepackage{steinmetz}
\usepackage{tikz}
\usepackage{circuitikz}
\usepackage{verbatim}
\usepackage{tfrupee}
\usepackage[breaklinks=true]{hyperref}
\usepackage{graphicx}
\usepackage{tkz-euclide}
\newcommand*{\Comb}[2]{{}^{#1}C_{#2}}%

\usetikzlibrary{calc,math}
\usepackage{listings}
    \usepackage{color}                                            %%
    \usepackage{array}                                            %%
    \usepackage{longtable}                                        %%
    \usepackage{calc}                                             %%
    \usepackage{multirow}                                         %%
    \usepackage{hhline}                                           %%
    \usepackage{ifthen}                                           %%
    \usepackage{lscape}     
\usepackage{multicol}
\usepackage{chngcntr}

\DeclareMathOperator*{\Res}{Res}

\renewcommand\thesection{\arabic{section}}
\renewcommand\thesubsection{\thesection.\arabic{subsection}}
\renewcommand\thesubsubsection{\thesubsection.\arabic{subsubsection}}

\renewcommand\thesectiondis{\arabic{section}}
\renewcommand\thesubsectiondis{\thesectiondis.\arabic{subsection}}
\renewcommand\thesubsubsectiondis{\thesubsectiondis.\arabic{subsubsection}}


\hyphenation{op-tical net-works semi-conduc-tor}
\def\inputGnumericTable{}                                 %%

\lstset{
%language=C,
frame=single, 
breaklines=true,
columns=fullflexible
}
\DeclareUnicodeCharacter{2212}{-}
\begin{document}

\newcommand{\BEQA}{\begin{eqnarray}}
\newcommand{\EEQA}{\end{eqnarray}}
\newcommand{\define}{\stackrel{\triangle}{=}}
\bibliographystyle{IEEEtran}
\raggedbottom
\setlength{\parindent}{0pt}
\providecommand{\mbf}{\mathbf}
\providecommand{\pr}[1]{\ensuremath{\Pr\left(#1\right)}}
\providecommand{\qfunc}[1]{\ensuremath{Q\left(#1\right)}}
\providecommand{\sbrak}[1]{\ensuremath{{}\left[#1\right]}}
\providecommand{\lsbrak}[1]{\ensuremath{{}\left[#1\right.}}
\providecommand{\rsbrak}[1]{\ensuremath{{}\left.#1\right]}}
\providecommand{\brak}[1]{\ensuremath{\left(#1\right)}}
\providecommand{\lbrak}[1]{\ensuremath{\left(#1\right.}}
\providecommand{\rbrak}[1]{\ensuremath{\left.#1\right)}}
\providecommand{\cbrak}[1]{\ensuremath{\left\{#1\right\}}}
\providecommand{\lcbrak}[1]{\ensuremath{\left\{#1\right.}}
\providecommand{\rcbrak}[1]{\ensuremath{\left.#1\right\}}}
\theoremstyle{remark}
\newtheorem{rem}{Remark}
\newcommand{\sgn}{\mathop{\mathrm{sgn}}}
\providecommand{\abs}[1]{\vert#1\vert}
\providecommand{\res}[1]{\Res\displaylimits_{#1}} 
\providecommand{\norm}[1]{\lVert#1\rVert}
%\providecommand{\norm}[1]{\lVert#1\rVert}
\providecommand{\mtx}[1]{\mathbf{#1}}
\providecommand{\mean}[1]{E[ #1 ]}
\providecommand{\fourier}{\overset{\mathcal{F}}{ \rightleftharpoons}}
%\providecommand{\hilbert}{\overset{\mathcal{H}}{ \rightleftharpoons}}
\providecommand{\system}{\overset{\mathcal{H}}{ \longleftrightarrow}}
	%\newcommand{\solution}[2]{\textbf{Solution:}{#1}}
\newcommand{\solution}{\noindent \textbf{Solution: }}
\newcommand{\cosec}{\,\text{cosec}\,}
\providecommand{\dec}[2]{\ensuremath{\overset{#1}{\underset{#2}{\gtrless}}}}
\newcommand{\myvec}[1]{\ensuremath{\begin{pmatrix}#1\end{pmatrix}}}
\newcommand{\mydet}[1]{\ensuremath{\begin{vmatrix}#1\end{vmatrix}}}
\numberwithin{equation}{subsection}
\makeatletter
\@addtoreset{figure}{problem}
\makeatother
\let\StandardTheFigure\thefigure
\let\vec\mathbf
\renewcommand{\thefigure}{\theproblem}
\def\putbox#1#2#3{\makebox[0in][l]{\makebox[#1][l]{}\raisebox{\baselineskip}[0in][0in]{\raisebox{#2}[0in][0in]{#3}}}}
     \def\rightbox#1{\makebox[0in][r]{#1}}
     \def\centbox#1{\makebox[0in]{#1}}
     \def\topbox#1{\raisebox{-\baselineskip}[0in][0in]{#1}}
     \def\midbox#1{\raisebox{-0.5\baselineskip}[0in][0in]{#1}}
\vspace{3cm}
\title{\textbf{AI1103-Assignment 7}}
\author{Name: Shashank Shanbhag, Roll Number: CS20BTECH11061 }
\maketitle
\newpage
\bigskip
\renewcommand{\thefigure}{\theenumi}
\renewcommand{\thetable}{\theenumi}

Download all python codes from
\begin{lstlisting}
https://github.com/SHASHANK-1-ALL/AI1103-ASSIGNMENT-7/blob/main/Assignment7.py
\end{lstlisting}
%
and latex-tikz codes from 
%
\begin{lstlisting}
https://github.com/SHASHANK-1-ALL/AI1103-ASSIGNMENT-7/blob/main/Assignment7.tex
\end{lstlisting}
\section*{\textbf{Question}}
(X,Y) follows bivariate normal distribution $N_2$(0,0,1,1,$\rho$),  -1 $<$ $\rho$ $<$ 1. Then,
\begin{enumerate}
    \item X+Y and X-Y are uncorrelated only if $\rho$ = 0
    \item X+Y and X-Y are uncorrelated only if $\rho$ $<$ 0
    \item X+Y and X-Y are uncorrelated only if $\rho$ $>$ 0
    \item X+Y and X-Y are uncorrelated for all values of $\rho$
\end{enumerate}

\section*{\textbf{Solution}}
Given that 
\begin{equation}
 \vec{M} = \myvec{ X \\ Y}
    \sim N \begin{bmatrix}
        \myvec{0 \\ 0},
        \begin{pmatrix}
            1 & \rho\\
            \rho & 1 
        \end{pmatrix}
    \end{bmatrix}
\end{equation}

Here, Mean matrix of X and Y is:
\begin{equation}
    \mu = \myvec{0 \\ 0}
\end{equation}
Covariance matrix of X and Y is:
\begin{equation}
\Sigma = \begin{pmatrix}
            1 & \rho\\
            \rho & 1 
        \end{pmatrix}
\end{equation}

Now X+Y and X-Y can be written as:
\begin{equation}
    X+Y = \myvec{1 \\ 1} ^\top
            \myvec{X \\ Y}
    = \vec{A^\top} \vec{M}
\end{equation}
\begin{equation}
    X-Y = \myvec{1 \\ -1} ^\top
            \myvec{X \\ Y}
    =\vec{B^\top} \vec{M}
\end{equation}
where
\begin{equation}
 \vec{A} = \myvec{1 \\ 1}
\end{equation}
and
\begin{equation}
    \vec{B} = \myvec{1 \\ -1}
\end{equation}

Defining Covariance in terms of expectation value:
\begin{align}
    Cov(X,Y)=& E[(X-\mu_x)(Y-\mu_y)] \\
    =& E[(X-0)(Y-0)]\\
    =& E(XY)
\end{align}
\begin{align}
 Cov(X+Y,X-Y) =& \vec{A^\top} \Sigma \vec{B} \\
    =& \myvec{1 \\ 1}^\top
    \begin{pmatrix}
            1 & \rho\\
            \rho & 1 
        \end{pmatrix}
        \myvec{1 \\ -1}\\ 
    =& \myvec{1+\rho \\ 1+\rho}^\top
        \myvec{1 \\ -1}\\
    =& (1+\rho)-1(1+\rho) \\
    =&   0
\end{align}

Note that 
\begin{align}
Var(X+Y) = Cov(X+Y , X+Y)\\ 
 Var(X-Y)  = Cov(X-Y , X-Y)
\end{align}
Hence,
\begin{align}
    Var(X+Y) =&\vec{A^\top} \Sigma \vec{A} \\
        =& \myvec{1 \\ 1}^\top
         \begin{pmatrix}
            1 & \rho\\
            \rho & 1 
        \end{pmatrix}
        \myvec{1 \\ 1}\\
    =& \myvec{1+\rho \\ 1+\rho}^\top
        \myvec{1 \\ 1}\\
    =& 1+\rho+1+\rho \\
    =& 2+2\rho \neq 0
\end{align}
\begin{align}
    Var(X-Y) =& \vec{B^\top} \Sigma \vec{B} \\
        =& \myvec{1 \\ -1}^\top
    \begin{pmatrix}
            1 & \rho\\
            \rho & 1 
        \end{pmatrix}
         \myvec{1 \\ -1}\\
    =& \myvec{1-\rho \\ \rho-1}^\top
        \myvec{1 \\ -1}\\
    =& 1-\rho-\rho+1 \\
    =& 2-2\rho \neq 0
\end{align}

So correlation coefficient is:
\begin{equation}
    \rho(X+Y,X-Y) = \frac{Cov(X+Y,X-Y)}{\sqrt{var(X+Y) \times var(X-Y)}}
    = 0
\end{equation}

$\implies$ X+Y and X-Y are uncorrelated irrespective of value of $\rho$ where $\rho \in \brak{-1,1}$.\\

$\therefore$ The correct answer is \textbf{option 4}.
\end{document}